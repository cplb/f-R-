\documentclass[aps,prd,amsfonts,amssymb,amsmath,nofootinbib,reprint,showpacs]{revtex4-1}
\usepackage{graphicx,color,psfrag}
\usepackage{mathrsfs}
\usepackage{dcolumn}
\usepackage{hyperref}

\newcommand{\eqnref}[1]{(\ref{eq:#1})}
\newcommand{\figref}[1]{fig.\ \ref{fig:#1}}
\newcommand{\Figref}[1]{Fig.\ \ref{fig:#1}}
\newcommand{\secref}[1]{Sec. \ref{sec:#1}}
\newcommand{\Tabref}[1]{Table \ref{tab:#1}}

\newcommand{\units}[1]{\ensuremath{~\mathrm{#1}}}

\newcommand{\sub}[1]{\ensuremath{_\text{#1}}}
\newcommand{\super}[1]{\ensuremath{^\text{#1}}}
\newcommand{\dd}{\ensuremath{\text{d}}}
\newcommand{\diff}[2]{\ensuremath{\frac{\dd {#1}}{\dd {#2}}}}
\newcommand{\difftwo}[2]{\ensuremath{\frac{\dd^2 {#1}}{\dd {#2}^2}}}
\newcommand{\intd}[4]{\ensuremath{\int_{#1}^{#2}{#3}\,\dd{#4}}}
\newcommand{\recip}[1]{\ensuremath{\frac{1}{#1}}}
\newcommand{\grad}{\ensuremath{\boldsymbol{\nabla}}}
\newcommand{\order}[1]{\ensuremath{\mathcal{O}({#1})}}

\begin{document}

%\preprint{}

\title{Linearized $f(R)$ Gravity: Gravitational Radiation \& Solar System Tests}

\author{Christopher P. L. Berry}
\email[]{cplb2@ast.cam.ac.uk}
\affiliation{Institute of Astronomy, Madingley Road, Cambridge, CB3 0HA, United Kingdom}

\date{\today}

\begin{abstract}
We investigate the linearized form of metric $f(R)$-gravity, assuming that $f(R)$ is analytic about $R = 0$ so it may be expanded as $f(R) = R + a_2R^2/2 + \ldots$. We show how gravitational radiation is modified in $f(R)$-gravity, admitting an extra mode of oscillation, that of the Ricci scalar, in addition to two transverse polarizations which are similar to their general relativistic counterparts. The energy-momentum pseudotensor for the gravitational radiation is derived. 
\end{abstract}

\pacs{}

\maketitle

\section{Introduction To $f(R)$ Theory}

General relativity (GR) is a well tested theory of gravity~\cite{Will2006}; however it is still exciting to explore alternative theories. This is motivated by the need to explain dark matter and dark energy in cosmology, trying to formulate a quantizable theory of gravity, or simple curiosity regarding the uniqueness of GR. One of the simplest extensions to standard GR is the class of $f(R)$ theories~\cite{Sotiriou2010, DeFelice2010}.

In this work we will look at metric $f(R)$-gravity. We focus on the modifications to gravitational radiation and possible solar system tests that can be used to constrain the theory. We begin with a review of review of the $f(R)$ field equations. Then in \secref{Lin} we derive the linearized equations. In \secref{Rad} we apply these to find wave solutions and in \secref{EM_tensor} we derive the energy-momentum pseudotensor for this gravitational radiation. In \secref{Source} we look at the effects of introducing a source term and derive the weak-field metrics for a point source, a rotating point source and a uniform density sphere. These are used in \secref{Tests} derive constraints for $f(R)$ based upon solar system and laboratory tests. Some results known in the literature are worked out here {\it ab initio}; they are included as a compendium of useful results, within a consistent system of notation, and to highlight some important points.

Throughout this work we will use the time-like sign convention of Landau and Lifshitz~\cite{Landau1975}:
\begin{enumerate}
\item The metric has signature $(+,-,-,-)$.
\item The Riemann tensor is defined as ${R^\mu}_{\nu\sigma\rho} = \partial_\sigma {\Gamma^\mu}_{\nu\rho} - \partial_\rho {\Gamma^\mu}_{\nu\sigma} + {\Gamma^\mu}_{\lambda\sigma}{\Gamma^\lambda}_{\rho\nu} - {\Gamma^\mu}_{\lambda\rho}{\Gamma^\lambda}_{\sigma\nu}$.
\item The Ricci tensor is defined as the contraction $R_{\mu\nu} = {R^\lambda}_{\mu\lambda\nu}$.
\end{enumerate}
Greek indices are used to represent spacetime indices $\mu = \{0,1,2,3\}$ and lowercase Latin indices from the middle of the alphabet are used for spatial indices $i = \{1,2,3\}$. Natural units with $c = 1$ will be used throughout, but factors of $G$ will be retained.

\subsection{The Action \& Field Equations\label{sec:Action}}

General relativity may be derived from the Einstein-Hilbert action~\cite{Misner1973, Landau1975}
\begin{equation}
S\sub{EH}[g] = \recip{16\pi G}\intd{}{}{R\sqrt{-g}}{^4x}.
\end{equation}
In $f(R)$ theory we make a simple modification of the action to include an arbitrary function of the Ricci scalar $R$ such that~\cite{Buchdahl1970}
\begin{equation}
S[g] = \recip{16\pi G}\intd{}{}{f(R)\sqrt{-g}}{^4x}.
\end{equation}
Including the function $f(R)$ gives extra freedom in defining the behaviour of gravity; while this action may not encode the true theory of gravity it could contain sufficient information to act as an effective field theory, correctly describing phenomenological behaviour~\cite{Park2010}. We will assume that $f(R)$ is analytic about $R = 0$ so that it can be expressed as a power series~\cite{Buchdahl1970, Psaltis2008}
\begin{equation}
f(R) = a_0 + a_1 R + \frac{a_2}{2!}R^2 + \frac{a_3}{3!}R^3 + \ldots
\end{equation}
Since the dimensions of $f(R)$ must be the same as of $R$, $[a_n] = [R]^{(1-n)}$. To link to GR we will set $a_1 = 1$; any rescaling can be absorbed into the definition of $G$.

The field equations are obtained by a variational principle; there are several ways of achieving this. To derive the Einstein field equations from the Einstein-Hilbert action one may use the standard metric variation or the Palatini variation~\cite{Misner1973}. Both approaches can be used for $f(R)$, however they yield different results~\cite{Sotiriou2010, DeFelice2010}. Following the metric formalism, one varies the action with respect to the metric $g^{\mu\nu}$, the resulting field equations being for metric $f(R)$-gravity. Following the Palatini formalism one varies the action with respect to both the metric $g^{\mu\nu}$ and the connection ${\Gamma^\rho}_{\mu\nu}$, which are treated as independent quantities: the connection is not the Levi-Civita metric connection.\footnote{Requiring that the metric and Palatini formalisms produce the same field equations, assuming an action that only depends on the metric and Riemann tensor, results in Lovelock gravity~\cite{Exirifard2008}. Lovelock gravities require the field equations to be divergence free and no more than second order; in four dimensions the only possible Lovelock gravity is GR with a potentially non-zero cosmological constant~\cite{Lovelock1970, Lovelock1971, Lovelock1972}.}

Finally, there is a third version of $f(R)$-gravity: metric-affine $f(R)$-gravity~\cite{Sotiriou2007, Sotiriou2007b}. This goes beyond the Palatini formalism by supposing that the matter action is dependent on the variational independent connection. Parallel transport and the covariant derivative are divorced from the metric. This theory has its attractions: it allows for a natural introduction of torsion. However, it is not a metric theory of gravity and so cannot satisfy all the postulates of the Einstein equivalence principle~\cite{Will2006}: a free particle does not necessarily follow a geodesic and so the effects of gravity might not be locally removed~\cite{Exirifard2008}. The implications of this have not been fully explored, but for this reason we shall not consider the theory further.

We shall restrict our attention to metric $f(R)$-gravity. This is preferred as the Palatini formalism has undesirable properties: static spherically symmetric objects described by a polytropic equation of state are subject to a curvature singularity~\cite{Barausse2008b, Barausse2008a}. Varying the action with respect to the metric $g^{\mu\nu}$ produces
\begin{equation}
\delta S = \recip{16\pi G}\intd{}{}{\left\{f'(R)\sqrt{-g}\left[R_{\mu\nu} - \nabla_\mu\nabla_\nu\ + g_{\mu\nu}\Box\right] - f(R)\frac{1}{2}\sqrt{-g}g_{\mu\nu}\right\}\delta g^{\mu\nu}}{^4x},
\end{equation}
where $\Box = g^{\mu\nu}\nabla_\mu\nabla_\nu$ is the d'Alembertian and a prime denotes differentiation with respect to $R$. Proceeding from here requires certain assumptions regarding surface terms. In the case of the Einstein-Hilbert action these gather into a total derivative. It is possible to subtract this from the action to obtain a well-defined variational quantity~\cite{York1972, Gibbons1977}. This is not the case for general $f(R)$~\cite{Madsen1989}. However, since the action includes higher-order derivatives of the metric we are at liberty to fix more degrees of freedom at the boundary, in so doing eliminating the importance of the surface terms~\cite{Dyer2009a, Sotiriou2010}. There is no well described prescription for this so we proceed directly to the field equations.

The vacuum field equations are
\begin{equation}
f'R_{\mu\nu} - \nabla_\mu\nabla_\nu f' + g_{\mu\nu}\Box f' - \frac{f}{2}g_{\mu\nu} = 0.
\label{eq:Field_eq}
\end{equation}
Taking the trace of our field equation gives
\begin{equation}
f'R + 3\Box f' - 2f = 0.
\label{eq:Trace_eq}
\end{equation}
If we consider a uniform flat spacetime $R = 0$, this equation gives
\begin{equation}
a_0 = 0.
\label{eq:a_0}
\end{equation}
In analogy to the Einstein tensor, we shall define
\begin{equation}
\mathcal{G}_{\mu\nu} = f'R_{\mu\nu} - \nabla_\mu\nabla_\nu f' + g_{\mu\nu}\Box f' - \frac{f}{2}g_{\mu\nu},
\label{eq:G_tensor}
\end{equation}
so that in a vacuum
\begin{equation}
\mathcal{G}_{\mu\nu} = 0.
\end{equation}

\subsection{Conservation Of Energy-Momentum}

If we introduce matter with a stress-energy tensor $T_{\mu\nu}$, the field equations become
\begin{equation}
\mathcal{G}_{\mu\nu} = 8\pi GT_{\mu\nu}.
\end{equation}
Acting upon this with the covariant derivative
\begin{eqnarray}
8\pi G\nabla^\mu T_{\mu\nu} & = & \nabla^\mu\mathcal{G}_{\mu\nu} \nonumber \\
& = & R_{\mu\nu}\nabla^\mu f' + f'\nabla^\mu\left(R_{\mu\nu} - \recip{2}R g_{\mu\nu}\right) - \left(\Box\nabla_\nu - \nabla_\nu\Box\right)f'.
\end{eqnarray}
The second term contains the covariant derivative of the Einstein tensor and so is zero. The final term can be shown to be
\begin{eqnarray}
\left(\Box\nabla_\nu - \nabla_\nu\Box\right)f' & = & g^{\mu\sigma}\left[\nabla_\mu\nabla_\sigma\nabla_\nu - \nabla_\nu\nabla_\mu\nabla_\sigma\right]f' \nonumber \\
 & = & R_{\tau\nu}\nabla^\tau f',
\end{eqnarray}
which is a useful geometric identity~\cite{Koivisto2006a}. Using this
\begin{eqnarray}
8\pi G\nabla^\mu T_{\mu\nu} & = & R_{\mu\nu}\nabla^\mu f' - R_{\mu\nu}\nabla^\mu f' \nonumber \\
 & = & 0.
\end{eqnarray}
Consequently energy-momentum is a conserved quantity in the same way as in GR, as is expected from the symmetries of the action.

\section{Linearized Theory\label{sec:Lin}}

We start our investigation of $f(R)$ by looking at linearized theory. This is a weak-field approximation that assumes only small deviations from a flat background, greatly simplifying the field equations. Just as in GR, the linearized framework provides a natural way to study gravitational waves. We will see that the linearized field equations will reduce down to flat-space wave equations: GWs are as much a part of $f(R)$-gravity as of GR.

Consider a perturbation of the metric from flat Minkowski space such that
\begin{equation}
g_{\mu\nu} = \eta_{\mu\nu} + h_{\mu\nu};
\end{equation}
where, more formally, we mean that $h_{\mu\nu} = \varepsilon H_{\mu\nu}$ for a small parameter $\varepsilon$.\footnote{It is because we wish to perturb about flat spacetime that we have required $f(R)$ to be analytic about $R = 0$.} We will consider terms only to $\order{\varepsilon}$. Thus, the inverse metric is
\begin{equation}
g^{\mu\nu} = \eta^{\mu\nu} - h^{\mu\nu},
\end{equation}
where we have used the Minkowski metric to raise the indices on the right, defining
\begin{equation}
h^{\mu\nu} = \eta^{\mu\sigma}\eta^{\nu\rho}h_{\sigma\rho}.
\end{equation}
Similarly, the trace $h$ is given by
\begin{equation}
h = \eta^{\mu\nu}h_{\mu\nu}.
\end{equation}
All quantities denoted by ``$h$'' are strictly $\order{\varepsilon}$.

The linearized connection is
\begin{equation}
{{\Gamma^{(1)}}^\rho}_{\mu\nu} = \frac{1}{2}\eta^{\rho\lambda}(\partial_\mu h_{\lambda\nu} + \partial_\nu h_{\lambda\mu} - \partial_\lambda h_{\mu\nu}).
\label{eq:Lin_Gamma}
\end{equation}
To $\order{\varepsilon}$ the covariant derivative of any perturbed quantity will be the same as the partial derivative. The Riemann tensor is
\begin{equation}
{{R^{(1)}}^\lambda}_{\mu\nu\rho} = \frac{1}{2}(\partial_\mu\partial_\nu h^\lambda_\rho + \partial^\lambda\partial_\rho h_{\mu\nu} - \partial_\mu\partial_\rho h^\lambda_\nu - \partial^\lambda\partial_\nu h_{\mu\rho}),
\label{eq:Lin_Riemann}
\end{equation}
where we have raised the index on the differential operator with the background Minkowski metric. Contracting gives the Ricci tensor
\begin{equation}
{R^{(1)}}_{\mu\nu} = \frac{1}{2}(\partial_\mu\partial_\rho h^\rho_\nu + \partial_\nu\partial_\rho h^\rho_\mu -\Box h_{\mu\nu} - \partial_\mu\partial_\nu h),
\label{eq:Ricci}
\end{equation}
where the d'Alembertian operator is $\Box = \eta^{\mu\nu}\partial_\mu\partial_\nu$. Contracting this with $\eta^{\mu\nu}$ gives the first order Ricci scalar
\begin{equation}
R^{(1)} = \partial_\mu\partial_\rho h^{\rho\mu} - \Box h.
\label{eq:Scalar}
\end{equation}

To $\order{\varepsilon}$ we can write $f(R)$ as a Maclaurin series
\begin{eqnarray}
f(R) & = & a_0 + R^{(1)};\\
f'(R) & = & 1 + a_2 R^{(1)}.
\end{eqnarray}
As we are perturbing from a Minkowski background where the Ricci scalar vanishes, we use \eqnref{a_0} to set $a_0 = 0$. Inserting these into \eqnref{G_tensor} and retaining terms to $\order{\varepsilon}$ yields
\begin{equation}
{\mathcal{G}^{(1)}}_{\mu\nu} = {R^{(1)}}_{\mu\nu} - \partial_\mu\partial_\nu(a_2 R^{(1)}) + \eta_{\mu\nu}\Box(a_2 R^{(1)}) - \frac{R^{(1)}}{2}\eta_{\mu\nu}.
\label{eq:Field}
\end{equation}
Now consider the linearized trace equation, from \eqnref{Trace_eq}
\begin{eqnarray}
\mathcal{G}^{(1)} & = & R^{(1)} + 3 \Box(a_2 R^{(1)}) - 2 R^{(1)} \nonumber \\
\mathcal{G}^{(1)} & = & 3a_2 \Box R^{(1)} - R^{(1)},
\label{eq:Box_R}
\end{eqnarray}
where $\mathcal{G}^{(1)} = \eta^{\mu\nu}{\mathcal{G}^{(1)}}_{\mu\nu}$. This is the massive inhomogeneous Klein-Gordon equation. Setting $\mathcal{G} = 0$, as for a vacuum, we obtain the standard Klein-Gordon equation
\begin{equation}
\Box R^{(1)} + \Upsilon^2 R^{(1)} = 0,
\end{equation}
defining the reciprocal length (squared)
\begin{equation}
\Upsilon^2 = -\recip{3a_2}.
\end{equation}
For a physically meaningful solution $\Upsilon^2 > 0$: we constrain $f(R)$ such that $a_2 < 0$~\cite{Schmidt1986, Teyssandier1990, Olmo2005c, Corda2007}. From $\Upsilon$ we define a reduced Compton wavelength
\begin{equation}
\lambdabar_R = \recip{\Upsilon},
\end{equation}
and mass
\begin{equation}
m_R = \hbar\Upsilon
\end{equation}
associated with this scalar mode.

The next step is to substitute in $h_{\mu\nu}$ to try to find wave solutions. We want a quantity $\overline{h}_{\mu\nu}$ that will satisfy a wave equation, related to $h_{\mu\nu}$ by
\begin{equation}
\overline{h}_{\mu\nu} = h_{\mu\nu} + A_{\mu\nu}.
\end{equation}
In GR we use the trace-reversed form where $A_{\mu\nu} = -(h/2)\eta_{\mu\nu}$. This will not suffice here, but let us look for a similar solution
\begin{equation}
\overline{h}_{\mu\nu} = h_{\mu\nu} - \frac{h}{2}\eta_{\mu\nu} + B_{\mu\nu}.
\end{equation}
The only rank two tensors in our theory are: $h_{\mu\nu}$, $\eta_{\mu\nu}$, ${R^{(1)}}_{\mu\nu}$, and $\partial_\mu\partial_\nu$; $h_{\mu\nu}$ has been used already, and we wish to eliminate ${R^{(1)}}_{\mu\nu}$, so we will try the simpler option based around $\eta_{\mu\nu}$. We want $B_{\mu\nu}$ to be $\order{\varepsilon}$. There are three scalar quantities that satisfy this: $h$, $R^{(1)}$ and $\Box R^{(1)}$; $h$ is used already and $\Box R^{(1)}$ is related to $R^{(1)}$ by \eqnref{Box_R}. Therefore, we construct an ansatz
\begin{equation}
\overline{h}_{\mu\nu} = h_{\mu\nu} + \left(b a_2 R^{(1)} - \frac{h}{2}\right)\eta_{\mu\nu},
\label{eq:Ansatz}
\end{equation}
where $a_2$ has been included to ensure dimensional consistency and $b$ is a dimensionless number. Contracting with the background metric yields
\begin{equation}
\overline{h} = 4b a_2 R^{(1)} - h,
\label{eq:h_trace}
\end{equation}
so we can eliminate $h$ in our definition of $\overline{h}_{\mu\nu}$ to give
\begin{equation}
h_{\mu\nu} = \overline{h}_{\mu\nu} + \left(b a_2 R^{(1)} -\frac{\overline{h}}{2}\right)\eta_{\mu\nu}.
\end{equation}
Just as in GR, we have the freedom to perform a gauge transformation~\cite{Misner1973, Hobson2006}: the field equations are gauge invariant since we started with a function of the gauge invariant Ricci scalar. We will assume a Lorenz, or de Donder, gauge choice so
\begin{equation}
\nabla^\mu \overline{h}_{\mu\nu} = 0;
\label{eq:Lorenz}
\end{equation}
to first order this is
\begin{equation}
\partial^\mu \overline{h}_{\mu\nu} = 0.
\end{equation}
Subject to this, from \eqnref{Ricci}, the Ricci tensor is
\begin{equation}
{R^{(1)}}_{\mu\nu} = -\frac{1}{2}\left\{2b a_2 \partial_\mu\partial_\nu R^{(1)} + \Box\left(\overline{h}_{\mu\nu} -\frac{\overline{h}}{2}\eta_{\mu\nu}\right) + \frac{b}{3}(R^{(1)} + \mathcal{G}^{(1)})\eta_{\mu\nu}\right\}.
\end{equation}
Using this with \eqnref{Box_R} in \eqnref{Field} gives
\begin{equation}
-\frac{1}{2}\Box\left(\overline{h}_{\mu\nu} - \frac{\overline{h}}{2}\right) - (b + 1)\left(a_2\partial_\mu\partial_\nu R^{(1)} + \reip{6}R^{(1)}}\eta_{\mu\nu}\right) = {\mathcal{G}^{(1)}}_{\mu\nu} + \frac{b-2}{6}\mathcal{G}^{(1)}\eta_{\mu\nu}.
\label{eq:b_Field}
\end{equation}
Picking $b = -1$ the second term vanishes, thus we will set~\cite{Corda2007, Capozziello2008}
\begin{eqnarray}
\overline{h}_{\mu\nu} & = & h_{\mu\nu} - \left(a_2 R^{(1)} + \frac{h}{2}\right)\eta_{\mu\nu}\\
h_{\mu\nu} & = & \overline{h}_{\mu\nu} - \left(a_2 R^{(1)} -\frac{\overline{h}}{2}\right)\eta_{\mu\nu}.
\label{eq:h_metric}
\end{eqnarray}
From \eqnref{Scalar} the Ricci scalar is 
\begin{eqnarray}
R^{(1)} & = & \Box \left(a_2 R^{(1)} -\frac{\overline{h}\eta_{\mu\nu}}{2}\right) - \Box (-4 a_2 R^{(1)} - \overline{h}) \nonumber \\
 & = & 3a_2 \Box R^{(1)} + \frac{1}{2}\Box \overline{h}.
\label{eq:Ricci_Box_h}
\end{eqnarray}
For consistency with \eqnref{Box_R}, we require
\begin{equation}
-\recip{2}\Box \overline{h} = \mathcal{G}^{(1)}.
\label{eq:Box_h}
\end{equation}
Inserting this into \eqnref{b_Field}, with $b = -1$, we see
\begin{equation}
-\recip{2}\Box \overline{h}_{\mu\nu} = {\mathcal{G}^{(1)}}_{\mu\nu};
\label{eq:Box_hmunu}
\end{equation}
we have our wave equation.

Should $a_2$ be sufficiently small that it can be regarded an $\order{\varepsilon}$ quantity, we recover GR to leading order within our analysis.

\section{Gravitational Radiation\label{sec:Rad}}

Having established two wave equations, \eqnref{Box_R} and \eqnref{Box_hmunu}, we now investigate their solutions. We consider waves in a vacuum, such that $\mathcal{G}_{\mu\nu} = 0$. Using a standard Fourier decomposition
\begin{eqnarray}
\overline{h}_{\mu\nu} & = & \widehat{h}_{\mu\nu}(k_\rho) \exp\left(ik_\rho x^\rho\right),\\
R^{(1)} & = & \widehat{R}(q_\rho) \exp\left(iq_\rho x^\rho\right),
\end{eqnarray}
where $k_\mu$ and $q_\mu$ are 4-wavevectors. From \eqnref{Box_hmunu} we know that $k_\mu$ is a null vector, so for a wave travelling along the $z$-axis
\begin{equation}
k^\mu = \omega(1, 0, 0, 1),
\end{equation}
where $\omega$ is the angular frequency. Similarly, from \eqnref{Box_R}
\begin{equation}
q^\mu = \left(\Omega, 0, 0, \sqrt{\Omega^2 - \Upsilon^2}\right),
\label{eq:Ricci_q}
\end{equation}
for frequency $\Omega$. These waves do not travel at $c$, but have a group velocity
\begin{equation}
v = \frac{\sqrt{\Omega^2 - \Upsilon^2}}{\Omega},
\end{equation}
provided that $\Upsilon^2 > 0$, $v < 1 = c$. For $\Omega < \Upsilon$, we will find an evanescently decaying wave instead of a propagating mode.

From the gauge condition \eqnref{Lorenz} we find that $k^\mu$ is orthogonal to $\widehat{h}_{\mu\nu}$,
\begin{equation}
k^\mu\widehat{h}_{\mu\nu} = 0,
\end{equation}
in this case
\begin{equation}
\widehat{h}_{0\nu} + \widehat{h}_{3\nu} = 0.
\label{eq:Transverse}
\end{equation}

Let us consider the implications of \eqnref{Box_h} using equations \eqnref{h_trace} and \eqnref{Box_R},
\begin{eqnarray}
\Box\left(4a_2R^{(1)} + h\right) & = & 0 \nonumber \\
\Box h & = & -\frac{4}{3}R^{(1)}.
\end{eqnarray}
For non-zero $R^{(1)}$ (as required for the Ricci mode) there is no way to make a gauge choice such that the trace $h$ will vanish~\cite{Corda2007, Capozziello2008}. This is distinct from in GR. It is possible, however, to make a gauge choice such that the trace $\overline{h}$ will vanish. Consider a gauge transformation generated by $\xi_\mu$ which satisfies $\Box \xi_\mu = 0$, and so has a Fourier decomposition
\begin{equation}
\xi_\mu = \widehat{\xi}_\mu \exp\left(ik_\rho x^\rho\right).
\end{equation}
A transformation
\begin{equation}
\overline{h}_{\mu\nu} \rightarrow \overline{h}_{\mu\nu} + \partial_\mu\xi_\nu + \partial_\nu\xi_\mu - \eta_{\mu\nu}\partial^\rho\xi_\rho,
\end{equation}
would ensure both conditions \eqnref{Lorenz} and \eqnref{Box_hmunu} are satisfied~\cite{Misner1973}. Under such a transformation
\begin{equation}
\widehat{h}_{\mu\nu} \rightarrow \widehat{h}_{\mu\nu} + i\left(k_\mu\widehat{\xi}_\nu + k_\nu\widehat{\xi}_\mu - \eta_{\mu\nu}k^\rho\widehat{\xi}_\rho\right).
\end{equation}
We may therefore impose four further constraints (one for each $\widehat{\xi}_\mu$) upon $\widehat{h}_{\mu\nu}$. We take these to be
\begin{equation}
\widehat{h}_{0\nu} = 0, \quad \widehat{h} = 0.
\end{equation}
This might appear to be five constraints, however we have already imposed \eqnref{Transverse}, and so setting $\widehat{h}_{00} = 0$ automatically implies $\widehat{h}_{03} = 0$. In this gauge we have
\begin{eqnarray}
h_{\mu\nu} & = & \overline{h}_{\mu\nu} - a_2 R^{(1)}\eta_{\mu\nu},\\
h & = & -4a_2R^{(1)}.
\label{eq:gauge}
\end{eqnarray}
Thus $\overline{h}_{\mu\nu}$ behaves just as its GR counterpart, so we define
\begin{equation}
\left[\widehat{h}_{\mu\nu}\right] =
\begin{bmatrix}
0 & 0 & 0 & 0\\
0 & h_+ & h_\times & 0\\
0 & h_\times & -h_+ & 0\\
0 & 0 & 0 & 0
\end{bmatrix},
\end{equation}
where $h_+$ and $h_\times$ are constants representing the amplitudes of the two transverse polarizations of gravitational radiation.

It is important that our solutions reduce to those of GR in the event that $f(R) = R$. In our linearized approach this corresponds to $a_2 \rightarrow 0$, $\Upsilon^2 \rightarrow \infty$. We see from \eqnref{Ricci_q} that in this limit it would take an infinite frequency to excite a propagating Ricci mode, and evanescent waves would decay away infinitely quickly. Therefore there would be no detectable Ricci modes and we would only observe the two polarizations found in GR. Additionally, $\overline{h}_{\mu\nu}$ would simplify to its usual trace-reversed form.

\section{Energy-momentum Tensor\label{sec:EM_tensor}}

We expect gravitational radiation to carry energy-momentum. Unfortunately, it is difficult to define a proper energy-momentum tensor for a gravitational field: as a consequence of the equivalence principle it is possible to transform to a freely falling frame, eliminating the gravitational field and any associated energy density at a given point, although we can still define curvature in the neighbourhood of this point~\cite{Misner1973, Hobson2006}. We will do nothing revolutionary here, but shall follow the approach of Isaacson~\cite{Isaacson1968, Isaacson1968a}. The full field equations \eqnref{Field_eq} have no energy-momentum tensor for the gravitational field on the right-hand side. However, by expanding beyond the linear terms we can find a suitable energy-momentum pseudotensor for gravitational radiation. Expanding $\mathcal{G}_{\mu\nu}$ in orders of $h_{\mu\nu}$
\begin{equation}
\mathcal{G}_{\mu\nu} = {\mathcal{G}^{(\text{B})}}_{\mu\nu} + {\mathcal{G}^{(1)}}_{\mu\nu} + {\mathcal{G}^{(2)}}_{\mu\nu} + \ldots
\label{eq:G_exp}
\end{equation}
We use $(\text{B})$ for the background term instead of $(0)$ to avoid potential confusion regarding its order in $\varepsilon$. So far we have assumed that our background is flat, however we can imagine that should the gravitational radiation carry energy-momentum then this would act as a source of curvature for the background. This is a second-order effect that may be encoded, to accuracy of $\order{\varepsilon^2}$, as
\begin{equation}
{\mathcal{G}^{(\text{B})}}_{\mu\nu} = -{\mathcal{G}^{(2)}}_{\mu\nu}.
\end{equation}
By shifting ${\mathcal{G}^{(2)}}_{\mu\nu}$ to the right-hand side we create an effective energy-momentum tensor. As in GR we will average over several wavelengths, assuming that the background curvature is on a larger scale~\cite{Misner1973},
\begin{equation}
{\mathcal{G}^{(\text{B})}}_{\mu\nu} = -\left\langle{\mathcal{G}^{(2)}}_{\mu\nu}\right\rangle.
\end{equation}
By averaging we probe the curvature in a macroscopic region about a given point in spacetime, yielding a gauge invariant measure of the gravitational field strength. The averaging can be thought of as smoothing out the rapidly varying ripples of the radiation, leaving only the coarse-grained component that acts as a source for the background curvature.\footnote{By averaging we do not try to localise the energy of a wave to within a wavelength; for the massive Ricci scalar mode we always consider scales greater than $\lambdabar_R$.} The energy-momentum pseudotensor for the radiation is
\begin{equation}
t_{\mu\nu} = -\recip{8\pi G}\left\langle{\mathcal{G}^{(\text{2})}}_{\mu\nu}\right\rangle.
\end{equation}

Having made this provisional identification, we must set about carefully evaluating the various terms in \eqnref{G_exp}. We begin as in \secref{Lin} by defining a total metric
\begin{equation}
g_{\mu\nu} = \gamma_{\mu\nu} + h_{\mu\nu},
\end{equation}
where $\gamma_{\mu\nu}$ is the background metric. This changes our definition for $h_{\mu\nu}$: instead of being the total perturbation from flat Minkowski, it is the dynamical part of the metric with which we associate radiative effects. Since we know that ${\mathcal{G}^{(\text{B})}}_{\mu\nu}$ is $\order{\varepsilon^2}$, we decompose our background metric as
\begin{equation}
\gamma_{\mu\nu} = \eta_{\mu\nu} + j_{\mu\nu},
\end{equation}
where $j_{\mu\nu}$ is $\order{\varepsilon^2}$ to ensure that ${{R^{(\text{B})}}^\lambda}_{\mu\nu\rho}$ is also $\order{\varepsilon^2}$. Therefore its introduction will make no difference to the linearized theory.

We will consider terms only to $\order{\varepsilon^2}$. We identify ${{\Gamma^{(1)}}^\rho}_{\mu\nu}$ from \eqnref{Lin_Gamma} to the accuracy of our analysis. There is one small subtlety: whether we use the background metric $\gamma^{\mu\nu}$ or $\eta^{\mu\nu}$ to raise indices of $\partial_\mu$ and $h_{\mu\nu}$. Fortunately, to the accuracy considered here, it does not make a difference; however, we will consider the indices to be changed with $\gamma^{\mu\nu}$. This is more appropriate for considering the effect of curvature on gravitational radiation. We will not distinguish between $\partial_\mu$ and ${\nabla^{(\text{B})}}_\mu$, the covariant derivative for the background metric: to the order of accuracy considered here covariant derivatives would commute and ${\nabla^{(\text{B})}}_\mu$ behaves just like $\partial_\mu$. The connection coefficient has
\begin{eqnarray}
{{\Gamma^{(1)}}^\rho}_{\mu\nu} & = & \frac{1}{2}\gamma^{\rho\lambda}\left[\partial_\mu \left(\overline{h}_{\lambda\nu} - a_2 R^{(1)}\gamma_{\lambda\nu}\right) + \partial_\nu \left(\overline{h}_{\lambda\mu} - a_2 R^{(1)}\gamma_{\lambda\mu}\right) \right. \nonumber \\
 & & - \left. \partial_\lambda \left(\overline{h}_{\mu\nu} - a_2 R^{(1)}\gamma_{\mu\nu}\right)\right],
\end{eqnarray}
and
\begin{eqnarray}
{{\Gamma^{(2)}}^\rho}_{\mu\nu} & = & -\frac{1}{2}h^{\rho\lambda}(\partial_\mu h_{\lambda\nu} + \partial_\nu h_{\lambda\mu} - \partial_\lambda h_{\mu\nu}) \nonumber \\
 & = & -\frac{1}{2}\left(\overline{h}^{\rho\lambda} - a_2 R^{(1)}\gamma^{\rho\lambda}\right)\left[\partial_\mu \left(\overline{h}_{\lambda\nu} - a_2 R^{(1)}\gamma_{\lambda\nu}\right) + \partial_\nu \left(\overline{h}_{\lambda\mu} - a_2 R^{(1)}\gamma_{\lambda\mu}\right) \right. \nonumber \\
 & & - \left. \partial_\lambda \left(\overline{h}_{\mu\nu} - a_2 R^{(1)}\gamma_{\mu\nu}\right)\right].
\end{eqnarray}
For the Ricci tensor we can use our linearized expression, \eqnref{Ricci}, for the first order term,
\begin{equation}
{R^{(1)}}_{\mu\nu} = 2 a_2\partial_\mu\partial_\nu R^{(1)} + \recip{6} R^{(1)}\gamma_{\mu\nu}.
\end{equation}
The next term is
\begin{eqnarray}
{R^{(2)}}_{\mu\nu} & = & \partial_\rho {{\Gamma^{(2)}}^\rho}_{\mu\nu} - \partial_\nu {{\Gamma^{(2)}}^\rho}_{\mu\rho} + {{\Gamma^{(1)}}^\rho}_{\mu\nu}{{\Gamma^{(1)}}^\sigma}_{\rho\sigma} - {{\Gamma^{(1)}}^\rho}_{\mu\sigma}{{\Gamma^{(1)}}^\sigma}_{\rho\nu} \nonumber \\
 & = & \frac{1}{2}\left\{\recip{2}\partial_\mu\overline{h}_{\sigma\rho}\partial_\nu\overline{h}^{\sigma\rho} + \overline{h}^{\sigma\rho}\left[\partial_\mu\partial_\nu\overline{h}_{\sigma\rho} + \partial_\sigma\partial_\rho\left(\overline{h}_{\mu\nu} - a_2 R^{(1)}\gamma_{\mu\nu}\right) \right.\right. \nonumber \\
 & & - \left.\left. \partial_\nu\partial_\rho\left(\overline{h}_{\sigma\mu} - a_2 R^{(1)} \gamma_{\sigma\mu}\right) - \partial_\mu\partial_\rho\left(\overline{h}_{\sigma\nu} - a_2 R^{(1)} \gamma_{\sigma\nu}\right)\right] \right. \nonumber \\
 & & + \left. \partial^\rho\overline{h}^\sigma_\nu\left(\partial_\rho\overline{h}_{\sigma\mu} - \partial_\sigma\overline{h}_{\rho\mu}\right) - a_2 \partial^\sigma R^{(1)}\partial_\sigma\overline{h}_{\mu\mu} + {a_2}^2 \left[2R^{(1)}\partial_\mu\partial_\nu R^{(1)} \right.\right. \nonumber \\
 & & + \left.\left. 3\partial_\mu R^{(1)}\partial_\nu R^{(1)} + R^{(1)} \Box^{(\text{B})} R^{(1)} \gamma_{\mu\nu}\right]\right\}.
\end{eqnarray}
The d'Alembertian is $\Box^{(\text{B})} = \gamma^{\mu\nu}\partial_\mu\partial_\nu$.

To find the Ricci scalar we contract the Ricci tensor with the full metric. To $\order{\varepsilon^2}$,
\begin{eqnarray}
R^{(\text{B})} & = & \gamma^{\mu\nu} {R^{(\text{B})}}_{\mu\nu} \\
R^{(1)} & = & \gamma^{\mu\nu} {R^{(1)}}_{\mu\nu} \\
R^{(2)} & = & \gamma^{\mu\nu} {R^{(2)}}_{\mu\nu} - h^{\mu\nu} {R^{(1)}}_{\mu\nu} \nonumber \\
 & = & \frac{3}{4}\partial_\mu\overline{h}_{\sigma\rho}\partial^\mu\overline{h}^{\sigma\rho} - \recip{2} \partial^\rho\overline{h}^{\sigma\mu}\partial_\sigma\overline{h}_{\rho\mu} - 2a_2 \overline{h}^{\mu\nu}\partial_\mu\partial_\nu R^{(1)} \nonumber \\
 & & + a_2 {R^{(1)}}^2 + \frac{3a_2}{2}\partial_\mu R^{(1)} \partial^\mu R^{(1)}.
\end{eqnarray}
Using these
\begin{eqnarray}
f^{(\text{B})} & = & R^{(\text{B})} \\
f^{(1)} & = & R^{(1)} \\
f^{(2)} & = & R^{(2)} + \frac{a_2}{2}{R^{(1)}}^2,
\end{eqnarray}
and
\begin{eqnarray}
f'^{(\text{B})} & = & a_2 R^{(\text{B})} \\
f'^{(0)} & = & 1 \\
f'^{(1)} & = & a_2 R^{(1)} \\
f'^{(2)} & = & a_2 R^{(2)} + \frac{a_3}{2}{R^{(1)}}^2.
\end{eqnarray}
We list a zeroth order term for clarity.

Combining all of these
\begin{eqnarray}
{\mathcal{G}^{(2)}}_{\mu\nu} & = & {R^{(2)}}_{\mu\nu} + f'^{(1)}{R^{(1)}}_{\mu\nu} - \partial_\mu\partial_\nu f'^{(2)} + {{\Gamma^{(1)}}^\rho}_{\nu\mu}\partial_\rho f'^{(1)} + \gamma_{\mu\nu}\gamma^{\rho\sigma}\partial_\rho\partial_\sigma f'^{(2)} \nonumber \\
 & & - \gamma_{\mu\nu}\gamma^{\rho\sigma}{{\Gamma^{(1)}}^\lambda}_{\sigma\rho}\partial_\lambda f'^{(1)} - \gamma_{\mu\nu}h^{\rho\sigma}\partial_\rho\partial_\sigma f'^{(1)} + h_{\mu\nu}\gamma^{\rho\sigma}\partial_\rho\partial_\sigma f'^{(1)} \nonumber \\
 & & - \recip{2}f^{(2)}\gamma_{\mu\nu} - \recip{2}f^{(1)}h_{\mu\nu} \nonumber \\
 & = & {R^{(2)}}_{\mu\nu} + a_2\left(\gamma_{\mu\nu}\Box^{(\text{B})} - \partial_\mu\partial_\nu\right)R^{(2)} - \recip{2}R^{(2)}\gamma_{\mu\nu} + a_3\left(\gamma_{\mu\nu}\Box^{(\text{B})} - \partial_\mu\partial_\nu\right){R^{(1)}}^2 \nonumber \\
 & & - \recip{6}\overline{h}_{\mu\nu}R^{(1)} - a_2\gamma_{\mu\nu}\overline{h}^{\sigma\rho}\partial_\sigma\partial_\rho R^{(1)} + \frac{a_2}{2} \partial^\rho R^{(1)} \left(\partial_\mu\overline{h}_{\rho\nu} + \partial_\nu\overline{h}_{\rho\mu} - \partial_\rho\overline{h}_{\mu\nu}\right) \nonumber \\
 & & + a_2\left(R^{(1)}{R^{(1)}}_{\mu\nu} + \recip{4}{R^{(1)}}^2\gamma_{\mu\nu}\right) - {a_2}^2\left(\partial_\mu R^{(1)}\partial_\nu R^{(1)} + \recip{2} \gamma_{\mu\nu}\partial^\rho R^{(1)}\partial_\rho R^{(1)}\right).
\end{eqnarray}
It is simplest to split this up for the purposes of averaging. Since we average over all directions at each point gradients average to zero~\cite{Hobson2006}
\begin{equation}
\left\langle\partial_\mu V\right\rangle = 0.
\end{equation}
As a corollary of this we have the relation
\begin{equation}
\left\langle U\partial_\mu V\right\rangle = -\left\langle V \partial_\mu U\right\rangle.
\end{equation}
Repeated application of this, together with our gauge condition, \eqnref{Lorenz}, and wave equations, \eqnref{Box_R} and \eqnref{Box_hmunu}, allows us to eliminate many terms. Terms that do not average to zero are
\begin{eqnarray}
\left\langle {R^{(2)}}_{\mu\nu} \right\rangle & = & \left\langle -\recip{4} \partial_\mu\overline{h}_{\sigma\rho}\partial^\mu\overline{h}^{\rho\sigma} + \frac{{a_2}^2}{2}\partial_\mu R^{(1)}\partial_\nu R^{(1)} + \frac{a_2}{6}\gamma_{\mu\nu}R^{(1)} \right\rangle; \\
\left\langle R^{(2)} \right\rangle & = & \left\langle \frac{3a_2}{2}{R^{(1)}}^2 \right\rangle; \\
\left\langle R^{(1)}{R^{(1)}}_{\mu\nu} \right\rangle & = & \left\langle a_2 R^{(1)} \partial_\mu\partial_\nu R^{(1)} + \recip{6}\gamma_{\mu\nu}{R^{(1)}}^2\right\rangle.
\end{eqnarray}
Combining these gives
\begin{equation}
\left\langle {\mathcal{G}^{(2)}}_{\mu\nu}\right\rangle = \left\langle -\recip{4} \partial_\mu\overline{h}_{\sigma\rho}\partial^\mu\overline{h}^{\rho\sigma} - \frac{3{a_2}^2}{2}\partial_\mu R^{(1)}\partial_\nu R^{(1)} \right\rangle.
\end{equation}
Thus we obtain the result
\begin{equation}
t_{\mu\nu} = \recip{32\pi G}\left\langle \partial_\mu\overline{h}_{\sigma\rho}\partial^\mu\overline{h}^{\rho\sigma} + 6{a_2}^2\partial_\mu R^{(1)}\partial_\nu R^{(1)} \right\rangle.
\end{equation}
In the limit of $a_2 \rightarrow 0$ we obtain the standard GR result as required. The GR result is also recovered if $R^{(1)} = 0$, as would be the case if the Ricci mode was not excited, for example if the frequency was below the cut off frequency $\Upsilon$. Rewriting the pseudotensor in terms of metric perturbation $h_{\mu\nu}$, using \eqnref{gauge},
\begin{equation}
t_{\mu\nu} = \recip{32\pi G}\left\langle \partial_\mu h_{\sigma\rho}\partial^\mu h^{\rho\sigma} + \recip{8}\partial_\mu h \partial_\nu h \right\rangle.
\end{equation}
These results do not depend upon $a_3$ or higher order coefficients.

These formulae could be used to constrain the parameter $a_2$ through observations of the energy and momentum carried away by gravitational radiation. Of particular interest would be a system with a frequency that evolved from $\omega < \Upsilon$ to $\omega > \Upsilon$ as then we would witness the switching on of the propagating Ricci mode. If we could accurate identify the cutoff frequency, we could accurately measure $a_2$. However, see \secref{Fifth} for further discussion of why this is unlikely to happen.

\section{$f(R)$ With A Source\label{sec:Source}}

Having considered radiation in a vacuum, we add a source term. We want a first order perturbation from the background metric so the linearized field equations are
\begin{equation}
{\mathcal{G}^{(1)}}_{\mu\nu} = 8\pi G T_{\mu\nu}.
\end{equation}
We will again assume a Minkowski background, considering terms to $\order{\varepsilon}$ only. To solve the wave equations \eqnref{Box_R} and \eqnref{Box_hmunu} with this source term we use a Green's function
\begin{equation}
\left(\Box + \Upsilon^2\right)\mathscr{G}_\Upsilon(x, x') = \delta(x - x'),
\end{equation}
where $\Box$ acts on $x$. The Green's function is familiar as the Klein-Gordon propagator (up to a factor of $-i$)~\cite{Peskin1995a}
\begin{equation}
\mathscr{G}_\Upsilon(x, x') = \int \frac{\dd^4 p}{(2\pi)^4} \frac{\exp\left[-ip\cdot(x-x')\right]}{\Upsilon^2 - p^2}.
\end{equation}
This can be evaluated by a suitable contour integral to give
\begin{equation}
\mathscr{G}_\Upsilon(x, x') =
\begin{cases}
{\displaystyle \int{\frac{\dd \omega}{2\pi} \exp\left[-i\omega(t-t')\right]\recip{4\pi r}\exp\left[i\left(\omega^2 - \Upsilon^2\right)^{1/2}r\right]}} & \omega^2 > \Upsilon^2\vspace{0.8mm}\\
{\displaystyle \int{\frac{\dd \omega}{2\pi} \exp\left[-i\omega(t-t')\right]\recip{4\pi r}\exp\left[-\left(\Upsilon^2 - \omega^2\right)^{1/2}r\right]}} & \omega^2 < \Upsilon^2\vspace{0.8mm}
\end{cases}\, ,
\label{eq:Green}
\end{equation}
where we have introduced $t = x^0$, $t' = x'^0$ and $r = |\boldsymbol{x} - \boldsymbol{x'}|$. For $\Upsilon = 0$
\begin{equation}
\mathscr{G}_0(x, x') = \frac{\delta(t - t' - r)}{4 \pi r},
\end{equation}
the familiar retarded-time Green's function. We can use this to solve \eqnref{Box_hmunu}
\begin{eqnarray}
\overline{h}_{\mu\nu}(x) & = & -16 \pi G \int \dd^4 x'\, \mathscr{G}_0(x, x') T_{\mu\nu}(x') \nonumber \\
 & = & -4 G \int \dd^3 x' \frac{T_{\mu\nu}(t - r, \boldsymbol{x'})}{r}.
\end{eqnarray}
This is exactly as in GR, so we can use standard results.

Solving for the scalar mode
\begin{equation}
R^{(1)}(x) = -8 \pi G \Upsilon^2 \int \dd^4 x'\, \mathscr{G}_\Upsilon(x, x') T(x').
\end{equation}
To proceed further we must know the form of the trace $T(x')$. In general the form of $R^{(1)}(x)$ will be complicated.

\subsection{The Newtonian Limit}

Let us consider the limiting case of a Newtonian source, such that
\begin{equation}
T_{00} = \rho; \quad |T_{00}| \gg |T_{0i}|; \quad |T_{00}| \gg |T_{ij}|,
\end{equation}
with a mass distribution of a stationary point source
\begin{equation}
\rho = M\delta(\boldsymbol{x'}).
\end{equation}
This source does not produce any radiation. As in GR
\begin{equation}
\overline{h}_{00} = -\frac{4GM}{r}; \quad \overline{h}_{0i} = \overline{h}_{ij} = 0.
\end{equation}
Solving for the Ricci scalar term gives
\begin{equation}
R^{(1)} = -2 G \Upsilon^2 M \frac{\exp(- \Upsilon r)}{r}.
\end{equation}
Combining these in \eqnref{h_metric} yields a metric perturbation with non-zero elements 
\begin{equation}
h_{00} = -\frac{2GM}{r}\left[1 + \frac{\exp(- \Upsilon r)}{3}\right]; \quad h_{ij} = -\frac{2GM}{r}\left[1 - \frac{\exp(- \Upsilon r)}{3}\right]\delta_{ij}.
\end{equation}
Thus, to first order, the metric for a point mass in $f(R)$-gravity is~\cite{Capozziello2009a, Naf2010}
\begin{eqnarray}
\dd s^2 & = & \left\{1-\frac{2GM}{r}\left[1 + \frac{\exp(- \Upsilon r)}{3}\right]\right\}\dd t^2 \nonumber \\
 & & - \left\{1+\frac{2GM}{r}\left[1 - \frac{\exp(- \Upsilon r)}{3}\right]\right\}\left(\dd x^2 + \dd y^2 + \dd z^2\right).
\label{eq:f(R)_Schw}
\end{eqnarray}
This is not the linearized limit of the Schwarzschild metric (although it is recovered as $a_2 \rightarrow 0$, $\Upsilon \rightarrow \infty$). Therefore Schwarzschild black holes (BHs) do not exist in $f(R)$-gravity (with $a_2 \neq 0$)~\cite{Chiba2007a}. This metric has already been derived for the case of quadratic gravity, which includes terms like $R^2$ and $R_{\mu\nu}R^{\mu\nu}$ in the Lagrangian~\cite{Pechlaner1966, Stelle1978, Schmidt1986, Teyssandier1990}. In linearized theory our $f(R)$ reduces to quadratic theory, as to first order $f(R) = R + a_2 R^2$.

Extending this result to a slowly rotating source with angular momentum $J$; we then have the additional term~\cite{Hobson2006}
\begin{equation}
\overline{h}^{0i} = -\frac{2G}{c^2r^3} \epsilon^{ijk}J_j x_k,
\end{equation}
where $\epsilon^{ijk}$ is the alternating Levi-Civita tensor. The metric is
\begin{eqnarray}
\dd s^2 & = & \left\{1-\frac{2GM}{r}\left[1 + \frac{\exp(- \Upsilon r)}{3}\right]\right\}\dd t^2 + \frac{4GJ}{r^3}\left(x\dd y - y\dd x\right)\dd t \nonumber \\ & & - {} \left\{1 +\frac{2GM}{r}\left[1 - \frac{\exp(- \Upsilon r)}{3}\right]\right\}\left(\dd x^2 + \dd y^2 + \dd z^2\right),
\end{eqnarray}
where $z$ is the rotation axis. This is not the first order limit of the Kerr metric (aside from in the limit $a_2 \rightarrow 0$, $\Upsilon \rightarrow \infty$).

It has been suggested that since $R = 0$ is a valid solution to the vacuum equations, the BH solutions of GR should also be solutions in $f(R)$~\cite{Psaltis2008, Barausse2008}. However we see that this is not the case: to have a BH you must have a source, and, because of \eqnref{Box_R}, this forces $R$ to be non-zero in the surrounding vacuum, although it will decay to zero at infinity~\cite{Olmo2007c}. BHs that exist in $f(R)$-gravity have a different structure than their GR counterparts. It should therefore be possible to distinguish between theories by observing the BHs that form.

Solving the full field equations to find the exact BH metric in $f(R)$ is difficult because of the higher-order derivatives that enter the equations. However, any solution must have the appropriate limiting form as given above.

In $f(R)$-gravity Birkhoff's theorem no longer applies: the metric about a spherically symmetric mass does not correspond to the equivalent of the Schwarzschild solution, since the distribution of matter influences how the Ricci scalar decays, and consequently Gauss' theorem no longer applies. Repeating our analysis for a (non-rotating) sphere of uniform density and radius $L$
\begin{equation}
\overline{h}_{00} = -\frac{4GM}{r}; \quad \overline{h}_{0i} = \overline{h}_{ij} = 0,
\end{equation}
as in GR, and for the point mass, but
\begin{eqnarray}
R^{(1)} & = & -6 G M \frac{\exp(- \Upsilon r)}{r}\left[\frac{\Upsilon L\cosh(\Upsilon L) - \sinh(\Upsilon L)}{\Upsilon L^3}\right] \\
 & = &  -6 G M \frac{\exp(- \Upsilon r)}{r}\Upsilon^2\Xi(\Upsilon L),
\end{eqnarray}
defining $\Xi(\Upsilon L)$ in the last line.\footnote{$\Xi(0) = 1/3$ is the minimum of $\Xi(\Upsilon L)$.} The metric perturbation thus has non-zero first order elements~\cite{Stelle1978, Capozziello2009b}
\begin{equation}
h_{00} = -2 G M \left[1 + \exp(- \Upsilon r)\Xi(\Upsilon L)\right]; \quad h_{ij} = -2 G M \left[1 - \exp(- \Upsilon r)\Xi(\Upsilon L)\right]\delta_{ij}.
\label{eq:Uniform}
\end{equation}
where we have assumed that $r > L$ at all stages.\footnote{Inside the source $R^{(1)} = -{6 G M}\left[1 - (\Upsilon L + 1)\exp(-\Upsilon L)\sinh(\Upsilon r)/\Upsilon r\right]/{L^3}$.}

In the next section we shall use these weak-field metrics with results from classic experimental tests of gravity to place constrints on $f(R)$.

\section{Solar System \& Laboratory Tests\label{sec:Tests}}

\subsection{Post-Newtonian Parameter $\gamma$}

The parameterized post-Newtonian (PPN) formalism was created to quantify deviations from GR~\cite{Will1993, Will2006}. It is ideal for solar system tests. The only parameter we need to consider here is $\gamma$ which measures the space-curvature produced by unit rest mass. The PPN metric has components
\begin{equation}
g_{00} = 1 + 2U; \quad g_{ij} = -(1 + 2\gamma U)\delta_{ij},
\end{equation}
where for a point mass
\begin{equation}
U(r) = \frac{GM}{r}.
\end{equation}
The $f(R)$ metric \eqnref{f(R)_Schw} is of a similar form, but there is not a direct correspondence because of the exponential. It has been suggested that this may be incorporated by changing the definition of the potential $U$~\cite{Olmo2007c, DeFelice2010}, then
\begin{equation}
\gamma = \frac{3 - \exp(-\Upsilon r)}{3 + \exp(-\Upsilon r)}.
\end{equation}
As $\Upsilon \rightarrow \infty$, the GR value of $\gamma = 1$ is recovered. However, the experimental bounds for $\gamma$ are derived assuming that it is a constant~\cite{Will1993}. Since this is not the case, we will rederive the post-Newtonian, or $\order{\varepsilon}$, corrections to photon trajectories for a more general metric. We will define
\begin{equation}
\dd s^2 = P(r)\dd t^2 - Q(r)\left(\dd x^2 + \dd y^2 + \dd z^2\right).
\end{equation}
To post-Newtonian order, this has non-zero connection coefficients
\begin{equation}
\begin{split}
{\Gamma^0}_{0i} = \frac{P'x^i}{2r}; \quad {\Gamma^i}_{00} = \frac{P'x^i}{2r}; \\
{\Gamma^i}_{jk} = \frac{Q'(\delta_{ij}x^k + \delta_{ik}x^j-\delta_{jk}x^i)}{2r}.
\end{split}
\end{equation}
The photon trajectory is described by the geodesic equation
\begin{equation}
\difftwo{x^\mu}{\sigma} + {\Gamma^\mu}_{\nu\rho}\diff{x^\nu}{\sigma}\diff{x^\rho}{\sigma} = 0,
\label{eq:Geodesic}
\end{equation}
for affine parameter $\sigma$. The time coordinate obeys
\begin{equation}
\difftwo{t}{\sigma} + {\Gamma^0}_{\nu\rho}\diff{x^\nu}{\sigma}\diff{x^\rho}{\sigma} = 0,
\end{equation}
so we can rewrite the spatial components of \eqnref{Geodesic} using $t$ as an affine parameter~\cite{Will1993}
\begin{equation}
\difftwo{x^i}{t} + \left({\Gamma^i}_{\nu\rho} - {\Gamma^0}_{\nu\rho}\diff{x^i}{t}\right)\diff{x^\nu}{t}\diff{x^\rho}{t} = 0.
\end{equation}
Since the geodesic is null we also have
\begin{equation}
g_{\mu\nu}\diff{x^\mu}{t}\diff{x^\nu}{t} = 0.
\end{equation}
To post-Newtonian accuracy these become
\begin{eqnarray}
\label{eq:Trajectory_1}
\difftwo{x^i}{t} & = & -\left(\frac{P'}{2r} - \frac{Q'}{2r}\left|\diff{\boldsymbol{x}}{t}\right|^2\right)x^i + \frac{P' - Q'}{r}\boldsymbol{x}\cdot\diff{\boldsymbol{x}}{t}\diff{x^i}{t}, \\
0 & = & P - Q\left|\diff{\boldsymbol{x}}{t}\right|^2.
\label{eq:Trajectory_2}
\end{eqnarray}
The Newtonian, or zeroth order, solution of these is straight-line propagation at constant speed~\cite{Will1993}
\begin{equation}
x^i\sub{N} = n^it; \quad |\boldsymbol{n}| = 1.
\end{equation}
The post-Newtonian trajectory can be written as
\begin{equation}
x^i = n^it + x^i\sub{pN}
\end{equation}
where $x^i\sub{pN}$ is the deviation from the straight-line trajectory. Substituting this into \eqnref{Trajectory_1} and \eqnref{Trajectory_2} gives
\begin{eqnarray}
\difftwo{\boldsymbol{x}\sub{pN}}{t} & = & -\recip{2}\grad(P - Q) + \boldsymbol{n}\cdot\grad(P - Q)\boldsymbol{n}, \\
\boldsymbol{n}\cdot\diff{\boldsymbol{x}\sub{pN}}{t} & = & \frac{P - Q}{2}.
\end{eqnarray}
The post-Newtonian deviation only depends upon the difference $P - Q$. From \eqnref{f(R)_Schw}
\begin{eqnarray}
P(r) - Q(r) & = & -\frac{4GM}{r} \nonumber \\
 & = & -4U(r).
\end{eqnarray}
This is identical to in GR. The result holds not just for a point mass, but for any stationary and static solution in $f(R)$. In these cases, using \eqnref{h_metric}
\begin{eqnarray}
P(r) - Q(r) & = & h_{00} + h_{ii} \quad \text{(no sum)}\nonumber \\
 & = & \overline{h}_{00} + \overline{h}_{ii},
\end{eqnarray}
and since $\overline{h}_{\mu\nu}$ obeys \eqnref{Box_hmunu} exactly as in GR, there is no difference. We conclude that an appropriate definition for the post-Newtonian parameter is
\begin{equation}
\gamma = \frac{g_{00} + g_{ii}}{2U} - 1 \quad \text{(no sum)}.
\end{equation}
Using this, our $f(R)$ solutions have $\gamma = 1$. Consequently, it is indistinguishable form GR in this respect and is entirely consistent with the current observational value of $\gamma = 1 + (2.1 \pm 2.3) \times 10^{-5}$~\cite{Will2006, Bertotti2003}.

\subsection{The Weak-Field Metric}

It is useful to transform the weak-field metric, \eqnref{f(R)_Schw}, to the more familiar form
\begin{equation}
\dd s^2 = A(\widetilde{r}) \dd t^2 - B(\widetilde{r})\dd \widetilde{r}^{\,2} - \widetilde{r}^{\,2} \dd \Omega^2.
\label{eq:Sph_sym}
\end{equation}
The coordinate $\widetilde{r}$ is a circumferential measure, as in the Schwarzschild metric, as opposed to $r$, used in preceding sections, which is a radial distance, an isotropic coordinate~\cite{Misner1973, Olmo2007c}. To simplify the algebra we shall introduce the Schwarzschild radius
\begin{equation}
r\sub{S} = 2GM.
\end{equation}
In the linearized regime, we require that the new radial coordinate satisfies
\begin{eqnarray}
\widetilde{r}^{\,2} & = & \left\{1 + \frac{r\sub{S}}{r}\left[1 - \frac{\exp(-\Upsilon r)}{3}\right]\right\}r^2 \\
\widetilde{r} & = & r + \frac{r\sub{S}}{2}\left[1 - \frac{\exp(-\Upsilon r)}{3}\right].
\label{eq:r_tilde}
\end{eqnarray}
To first order in ${r\sub{S}}/{r}$~\cite{Olmo2007c}
\begin{equation}
A(\widetilde{r}) = 1 - \frac{r\sub{S}}{\widetilde{r}}\left[1 + \frac{\exp(-\Upsilon r )}{3}\right].
\label{eq:A_metric}
\end{equation}
We see that the functional form of $g_{00}$ is almost unchanged upon substituting $\widetilde{r}$ for $r$; however $r$ is still in the exponential.

To find $B(\widetilde{r})$ we consider, using \eqnref{r_tilde},
\begin{eqnarray}
\frac{\dd \widetilde{r}}{\widetilde{r}} & = & \dd \ln \widetilde{r} \nonumber \\
 & = & \left\{\frac{1 + {\Upsilon r\sub{S}r\exp(-\Upsilon r)}/{6\widetilde{r}}}{1 + ({r\sub{S}}/{2\widetilde{r}})\left[1 - {\exp(-\Upsilon \widetilde{r})}/{3}\right]}\right\}\frac{\dd r}{\widetilde{r}}.
\end{eqnarray}
Thus
\begin{equation}
\dd \widetilde{r}^{\,2} = \frac{\widetilde{r}^{\,2}}{r^2}\left\{\frac{1 + {\Upsilon r\sub{S}r\exp(-\Upsilon r)}/{6\widetilde{r}}}{1 + ({r\sub{S}}/{2\widetilde{r}})\left[1 - {\exp(-\Upsilon r)}/{3}\right]}\right\}\dd r^2.
\end{equation}
The term in braces is $\left[B(\widetilde{r})\right]^{-1}$. We assume that in the weak-field
\begin{equation}
\varepsilon \sim \frac{r\sub{S}}{r}
\end{equation}
is small. Then the metric perturbations from Minkowski are small. Expanding to first order~\cite{Olmo2007c}
\begin{equation}
B(\widetilde{r})  = 1 + \frac{r\sub{S}}{\widetilde{r}}\left[1 + \frac{\exp(-\Upsilon r )}{3}\right] - \frac{\Upsilon r\sub{S} \exp(-\Upsilon r\sub{S})}{3}.
\label{eq:B_metric}
\end{equation}
In the limit $\Upsilon \rightarrow \infty$, where we recover GR, $A(\widetilde{r})$ and $B(\widetilde{r})$ tend to their Schwarzschild forms.

\subsection{Epicyclic Frequencies\label{sec:Epicycle}}

One means of probing the nature of a spacetime is through observations of orbital motions~\cite{Gair2008a}. We will consider the epicyclic motion produced by perturbing a circular orbit. We will start by deriving a general result for any metric of the form of \eqnref{Sph_sym}, and then use this for our $f(R)$ solution.

An orbit in a spacetime described by \eqnref{Sph_sym} has three constants of motion: the orbiting particle's rest mass $\mu$, the energy (per unit mass) of the orbit $E$, and the $z$-component of the angular momentum (per unit mass) $L$. Using an over-dot to denote differentiation with respect to an affine parameter, which we shall identify as proper time $\tau$,
\begin{eqnarray}
E & = & A\dot{t}; \\
L & = & \widetilde{r}^{\,2}\sin^2\theta\, \dot{\phi}.
\end{eqnarray}
As a consequence of the spherical symmetry we can confine the motion to the equatorial plane $\theta = \pi/2$ without loss of generality. From the Hamiltonian $\mathcal{H} = g_{\mu\nu}\dot{x}^\mu\dot{x}^\nu$ we obtain the equation of motion for massive particles
\begin{equation}
\dot{\widetilde{r}}^{\,2} = \frac{E^2}{AB} - \recip{B}\left(1 + \frac{L^2}{\widetilde{r}^{\,2}}\right).
\label{eq:rdot}
\end{equation}
Hence for a circular orbit
\begin{equation}
E^2 = A\left(1 + \frac{L^2}{\widetilde{r}^{\,2}}\right).
\end{equation}
Differentiating \eqnref{rdot} yields
\begin{equation}
\ddot{\widetilde{r}} = -\frac{E^2}{2AB}\left(\frac{A'}{A} + \frac{B'}{B}\right) + \frac{B'}{2B^2}\left(1 + \frac{L^2}{\widetilde{r}^{\,2}}\right) + \frac{L^2}{\widetilde{r}^{\,3}B},
\label{eq:geodesic}
\end{equation}
where a prime signifies differentiation with respect to $\widetilde{r}$. For a circular orbit
\begin{equation}
0 = \frac{2L^2}{\widetilde{r}^{\,3}} - \frac{A'}{A}\left(1 + \frac{L^2}{\widetilde{r}^{\,2}}\right).
\end{equation}
Thus a circular orbit is defined by one of $\{E,L,\widetilde{r}\}$. We will consider a small perturbation to a circular orbit. Perturbations out of the plane just redefine the orbital plane; they are not of interest. A radial perturbation can be parameterized as
\begin{equation}
\widetilde{r} = \overline{r} + \delta,
\end{equation}
where $\overline{r}$ is the radius of the unperturbed orbit. We will denote $A(\overline{r}) = \overline{A}$ and $B(\overline{r}) = \overline{B}$. Substituting into \eqnref{geodesic} and retaining terms to first order
\begin{equation}
\ddot{\delta} = - \frac{2\overline{A}^2L^2}{\overline{r}^3\overline{A}'\overline{B}}\left(\frac{\overline{A}''}{2\overline{A}^2} - \frac{{\overline{A}'}^2}{\overline{A}^3}\right)\delta + \frac{3L^2}{\overline{r}^4}\delta.
\end{equation}
Assuming a solution of form $\delta = \delta_0\cos(-i\Omega\tau)$,
\begin{equation}
\Omega^2 = \frac{L^2}{\overline{r}^3\overline{B}}\left(\frac{\overline{A}''}{\overline{A}'} - \frac{2\overline{A}'}{\overline{A}} + \frac{3}{\overline{r}}\right).
\end{equation}
We will rewrite the radial motion as
\begin{equation}
\widetilde{r} = \overline{r} + \delta_0\cos(-i\Omega\tau).
\end{equation}
If we compare this with an elliptic Keplerian orbit of small eccentricity $e$
\begin{eqnarray}
\widetilde{r} & = & \frac{a(1 - e^2)}{1 + e\cos(\omega_0\tau)} \\
 & = & a\left[1 - e\cos(\omega_0\tau) + \ldots \, \right]
\end{eqnarray}
to first order in $e$, where $a$ is the semimajor axis and $\omega_0$ is the orbital frequency; we can identify the perturbed orbit with an elliptical orbit where~\cite{Kerner2001a}
\begin{equation}
\overline{r} = a; \quad \delta_0 = -ea.
\end{equation}
The eccentricity is the small parameter $|e| = |\delta_0/\overline{r}| \ll 1$. To this accuracy one cannot distinguish between $a$ and the semilatus rectum $p = a(1 - e^2)$.

Unless $\omega_0 = \Omega$ the elliptical motion will be asynchronous with the orbital motion: there will be precession of the periapsis. The orbital frequency is
\begin{equation}
\omega_0^2 = \frac{L^2}{\overline{r}^4}.
\end{equation}
In one revolution the ellipse will precess about the focus by
\begin{eqnarray}
\varpi & = & \omega_0\left(\frac{2\pi}{\Omega} - \frac{2\pi}{\omega_0}\right) \nonumber \\
 & = & 2\pi\left(\frac{\omega_0}{\Omega} - 1\right)
\end{eqnarray}
The precession is cumulative, so a small deviation may be measurable over sufficient time.

For the $f(R)$ metric defined by equations \eqnref{A_metric} and \eqnref{B_metric} the epicyclic frequency is
\begin{equation}
\Omega^2 = \omega_0^2 \left[1 - \frac{3r\sub{S}}{\overline{r}} - \zeta(\Upsilon,r\sub{S},\overline{r})\right],
\end{equation}
defining the function
\begin{eqnarray}
\zeta(\Upsilon,r\sub{S},\overline{r}) & = & r\sub{S}\left(\recip{3\overline{r}} + \Upsilon\right)\exp(-\Upsilon r) + \frac{\Upsilon^2\overline{r}^2\exp(-\Upsilon r)}{3 + (1 + \Upsilon \overline{r})\exp(-\Upsilon r)} \nonumber \\
& &  \times \left\{1 - \frac{r\sub{S}}{\overline{r}}\left[1 + \frac{\exp(-\Upsilon r)}{3}\right] - \frac{\Upsilon r\sub{S}\exp(-\Upsilon r)}{3}\right\}.
\end{eqnarray}
This characterizes the deviation from the Schwarzschild case: the change in the precession per orbit relative to Schwarzschild is
\begin{eqnarray}
\Delta \varpi & = & \varpi - \varpi\sub{S} \\
 & = & \pi\zeta,
\end{eqnarray}
using the subscript $\text{S}$ to denote the Schwarzschild value. To obtain the last line we have expanded to lowest order, assuming that $\zeta$ is small.\footnote{There is one term in $\zeta$ that is not explicitly $\order{\varepsilon}$. Numerical evaluation shows that this is $< 0.6$ for the applicable range of parameters.} Since $\zeta \geq 0$, the precession rate is enhanced relative to GR.

We can apply this to the classic test of planetary precession in the solar system. \Tabref{Precess} shows the orbital properties of the planets. We will use the deviation in perihelion precession rate from the GR prediction to constrain the value of $\zeta$, and hence $\Upsilon$ and $a_2$.
\begingroup
\squeezetable
\begin{table*}
\begin{ruledtabular}
\begin{tabular}{l D{.}{.}{2.8} D{.}{.}{3.8} c D{.}{.}{1.8}}
 & \multicolumn{1}{c}{Semimajor axis~\cite{Cox2000}} & \multicolumn{1}{c}{Orbital period~\cite{Cox2000}} & Precession rate~\cite{Pitjeva2009a} & \multicolumn{1}{c}{Eccentricity~\cite{Cox2000}} \\
Planet & \multicolumn{1}{c}{$r/10^{11}\units{m}$} & \multicolumn{1}{c}{$(2\pi/\omega_0)/mathrm{yr}$} &$\Delta \varpi \pm \sigma_{\Delta \varpi}/\mathrm{mas\,yr^{-1}}$ & \multicolumn{1}{c}{$e$} \\
\hline
Mercury & 0.57909175 & 0.24084445 & $\phantom{0}{-0.040} \pm \phantom{0}0.050\phantom{0}$ & 0.20563069 \\
Venus & 1.0820893 & 0.61518257 & $\phantom{-0}0.24\phantom{0} \pm \phantom{0}0.33\phantom{00}$ & 0.00677323 \\
Earth & 1.4959789 & 0.99997862 & $\phantom{-0}0.06\phantom{0} \pm \phantom{0}0.07\phantom{00}$ & 0.01671022 \\
Mars & 2.2793664 & 1.88071105 & $\phantom{0}{-0.07}\phantom{0} \pm \phantom{0}0.07\phantom{00}$ & 0.09341233 \\
Jupiter & 7.7841202 & 11.85652502 & $\phantom{-0}0.67\phantom{0} \pm \phantom{0}0.93\phantom{00}$ & 0.04839266 \\
Saturn & 14.267254 & 29.42351935 & $\phantom{0}{-0.10}\phantom{0} \pm \phantom{0}0.15\phantom{00}$ & 0.05415060 \\
Uranus & 28.709722 & 83.74740682 & ${-38.9}\phantom{00} \pm 39.0\phantom{000}$ & 0.04716771 \\
Neptune & 44.982529 & 163.723204 & ${-44.4}\phantom{00} \pm 54.0\phantom{000}$ & 0.00858587 \\
Pluto & 59.063762 & 248.0208 & $\phantom{-}28.4\phantom{00} \pm 25.1\phantom{000}$ & 0.24880766 \\
\end{tabular}
\end{ruledtabular}
\caption{Orbital properties of the eight major planets and Pluto. We take the semimajor orbital axis to be the flat-space distance $r$, not the coordinate $\widetilde{r}$. The eccentricity is not used in calculations, but is given to assess the accuracy of neglecting terms $\order{e^2}$.\label{tab:Precess}}
\end{table*}
\endgroup
Since several of the deviations are negative, they cannot be explained by $f(R)$ corrections. This may be considered as evidence against $f(R)$-gravity; however, all the precession rates are consistent with GR predictions ($\Delta \varpi = 0$), thus we cannot conclusively rule out $f(R)$-gravity. Since the deviations are zero to within their uncertainties, we may use the size of these uncertainties to constrain the $f(R)$ correction. \Tabref{Constraint} shows the constraints for $\Upsilon$ and $a_2$ obtained by equating the uncertainty in the precession rate $\sigma_{\Delta \varpi}$ with the $f(R)$ correction, and similarly using twice the uncertainty $2\sigma_{\Delta \varpi}$.
\begingroup
\squeezetable
\begin{table}
\begin{ruledtabular}
\begin{tabular}{l D{.}{.}{2.2} D{.}{.}{5.1} D{.}{.}{2.2} D{.}{.}{5.1}}
 &  \multicolumn{2}{c}{Using $\sigma_{\Delta \varpi}$} & \multicolumn{2}{c}{Using $2\sigma_{\Delta \varpi}$} \\
Planet & \multicolumn{1}{c}{$\Upsilon/10^{-11}\units{m^{-1}}$} & \multicolumn{1}{c}{$|a_2|/10^{18}{\metre^2}$} & \multicolumn{1}{c}{$\Upsilon/10^{-11}\units{m^{-1}}$} & \multicolumn{1}{c}{$|a_2|/10^{18}\units{m^2}$} \\
\hline
Mercury & 52.6 & 1.2 & 51.3 & 1.3 \\
Venus & 25.3 & 5.2 & 24.6 & 5.5 \\
Earth & 19.1 & 9.1 & 18.6 & 9.6 \\
Mars & 12.2 & 22 & 11.9 & 24 \\
Jupiter & 2.96 & 380 & 2.87 & 410 \\
Saturn & 1.69 & 1200 & 1.63 & 1200 \\
Uranus & 0.58 & 9800 &  0.56 & 11000 \\
Neptune & 0.35 & 28000 & 0.33 & 31000 \\
Pluto & 0.26 & 49000 & 0.25 & 55000 \\
\end{tabular}
\end{ruledtabular}
\caption{Bounds calculated using uncertainties in planetary perihelion precession rates. $\Upsilon$ must be greater than or equal to the tabulated value, $|a_2|$ must be less than or equal to the tabulated value.\label{tab:Constraint}}
\end{table}
\endgroup

The tightest constraint is obtained from the orbit of Mercury. Adopting a value of $\Upsilon \geq 5.26 \times 10^{10}\units{m^{-1}}$, the cutoff frequency for the Ricci mode is $\geq 0.158\units{s^{-1}}$. Therefore it could lie in the upper range of the LISA frequency band~\cite{Bender1998,Danzmann2003} or in the LIGO/Virgo frequency range~\cite{Abramovici1992, Abbott2009, Accadia2010}. However, as we will see in \secref{Fifth}, it is possible to place stronger constriants on $\Upsilon$ using laboratory experiments.

\subsection{Fifth-Force Tests\label{sec:Fifth}}

From the metric \eqnref{f(R)_Schw} we see that a point mass has a Yukawa gravitational potential~\cite{Stelle1978, Capozziello2009a}
\begin{equation}
V(r) = \frac{GM}{r}\left[1 + \frac{\exp(- \Upsilon r)}{3}\right].
\end{equation}
Potentials of this form are well studied in fifth-force tests~\cite{Will2006, Adelberger2009, Adelberger2003} which consider a potential defined by a coupling constant $\alpha$ and a length-scale $\lambdabar$ such that
\begin{equation}
V(r) = \frac{GM}{r}\left[1 + \alpha\exp\left(-\frac{r}{\lambdabar}\right)\right].
\end{equation}
We are able to put strict constraints upon our length-scale $\lambdabar_R$, and hence $a_2$, since our coupling constant $\alpha_R = 1/3$ is relatively large. This should be larger for extended sources: comparison with \eqnref{Uniform} shows that for a uniform sphere $\alpha_R = \Xi(\Upsilon L) \geq 1/3$.

The best constraints at short distances come from the E\"{o}t-Wash experiments, which use torsion balances~\cite{Kapner2007a, Hoyle2004}. These constrain $\lambdabar_R \lesssim 8\times 10^{-5}\units{m}$. Hence we determine $|a_2| \lesssim 2 \times 10^{-9}\units{m^2}$. A similar result is obtained by N\"{a}f and Jetzer~\cite{Naf2010}. This would mean that the cutoff frequency for a propagating scalar mode would be $\gtrsim 4 \times 10^{12}\units{s^{-1}}$. This is much higher than expected for astrophysical objects.

Fifth-force tests also permit $\lambdabar_R$ to be large. This degeneracy can be broken using other tests, from \secref{Epicycle} we know that the large range for $\lambdabar_R$ is excluded by planetary precession rates.

While the laboratory bound on $\lambdabar_R$ may be strict compared to astronomical length-scales, it is still much greater than the expected characteristic gravitational scale, the Planck length $\ell\sub{P}$. We might expect for a natural quantum theory, that $a_2 \sim \order{\ell\sub{P}^2}$; however $\ell\sub{P}^2 = 2.612 \times 10^{-70}{m^2}$, thus the bound is still about $60$ orders of magnitude greater than the natural value. The only other length-scale that we could introduce would be defined by the cosmological constant $\Lambda$. Using the concordance values~\cite{Hinshaw2009} $\Lambda = 1.27 \times 10^{-52}\units{m^{-2}}$; we see that $\Lambda^{-1} \gg |a_2|$. It is intriguing that if we combine these two length-scales we find ${\ell\sub{P}}/{\Lambda^{1/2}} = 1.44 \times 10^{-9}\units{m^2}$, which is on the order of the current bound. This is likely to be a coincidence, since there is nothing fundamental about the current level of precision. It would be interesting to see if the measurements could be improved to rule out a Yukawa interaction around this length-scale.

\section{Summary \& Conclusions\label{sec:f_Discuss}}

We have seen that gravitational radiation is modified in $f(R)$-gravity, as the Ricci scalar is no longer constrained to be zero. In linearized theory we discover that there is an additional mode of oscillation, that of the Ricci scalar. This is only excited above a cutoff frequency; once a propagated mode is excited, it will carry additional energy-momentum away from the source. In $f(R)$ theory, the two transverse GW modes are modified from their GR counterparts to include a contribution from the Ricci scalar, see \eqnref{h_metric}, allowing us to probe the curvature of the strong-field regions from which GWs originate. However, further study is needed in order to understand how GW waves behave in a region with background curvature, in particular when $R$ is non-zero.

Gravitational radiation is not the only way to test $f(R)$ theory. From linearized theory we have deduced the weak-field metrics for some simple mass distributions. These indicate that BH solutions are not the same as in GR. Using these weak-field results it is possible to constrain the value of $a_2$. The strongest constraints come from fifth-force tests, but we have also derived the epicyclic frequency for near circular orbits. Some of the estimated deviations from GR precession rates are negative, which cannot be achieved with $f(R)$ corrections. Since all of the deviations are consistent with zero, we cannot use these as proof against $f(R)$, just that it does not modify gravity on solar system scales.

Based upon the results of fifth-force tests, we find that $|a_2| \lesssim 2 \times 106{-9}\units{m^2}$. In this case we do not expect the propagating Ricci mode to be excited by astrophysical systems as the cutoff frequency is too high. Even in the absence of the Ricci mode, it may still be possible to use gravitational radiation to constrain $f(R)$-gravity through the dependence of the transverse polarization's dependence upon the Ricci scalar.

We have only discussed tests in the solar system; we must therefore be cautious with the range of applicability of our results. It is possible that $f(R)$-gravity is not universal: that it is different in different regions of space. This could occur if $f(R)$ is just an approximate effective theory, then the range of a particular parametrization's applicability could be limited to a specific domain. For example, we could imagine that the effective theory in the vicinity of a massive BH where the curvature is large is different from in the solar system where curvature is small; alternatively $f(R)$ could evolve with cosmological epoch so that it varies with redshift.

Another possibility is that $f(R)$-gravity is modified in the presence of matter via the chameleon mechanism~\cite{Khoury2004, Khoury2004a}. In metric $f(R)$ this corresponds to a nonlinear effect arising from a large departure of the Ricci scalar from its background value~\cite{DeFelice2010}. The mass of the effective scalar degree of freedom then depends upon the density of its environment. In a region of high matter density, such as the Earth, the deviations from standard gravity would be exponentially suppressed due to a large effective $\Upsilon$; while on cosmological scales, where the density is low, the scalar would have a small $\Upsilon$, perhaps of the order $H_0/c$~\cite{Khoury2004, Khoury2004a}. The chameleon mechanism allows $f(R)$ gravity to pass solar system tests while remaining of interest for cosmology. In the context of gravitational radiation, this would mean that the Ricci scalar mode could freely propagate on cosmological scales~\cite{Corda2009}. Unfortunately, since the chameleon mechanism suppresses the effects of $f(R)$ in the presence of matter, this mode would have to be excited by something other than the movement of matter. Additionally since electromagnetic radiation and the standard transverse polarizations of gravitational radiation have a traceless energy-momentum tensor, they cannot excite the Ricci mode.\footnote{The contribution to the gravitational energy-momentum pseudotensor from a propagating Ricci mode does have a non-zero trace.} To be able to detect the Ricci mode we must observe it well away from any matter, which would cause it to become evanescent: a spaceborne detector such as LISA would be our only hope.

An interesting extension to the work presented here is to consider the case when $a_0$ is non-zero. We could then consider an expansion about (anti-)de Sitter space. This is relevant because the current $\Lambda$CDM paradigm indicates that we live in a universe with a positive cosmological constant~\cite{Hinshaw2009}. Such a study would naturally compliment an investigation into the effects of background curvature on propagation.

\begin{acknowledgments}
Thanks to Jonathan Gair for suggesting this research topic and for his continued guidance and encouragement. The author is supported by an STFC studentship.
\end{acknowledgments}

\bibliography{../library}

\end{document}
