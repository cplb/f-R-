\documentclass[aps,prd,amsfonts,amssymb,amsmath,nofootinbib,reprint,showpacs]{revtex4-1}
\usepackage{graphicx,color,psfrag}
\usepackage{mathrsfs}
\usepackage{dcolumn}
\usepackage{hyperref}

\newcommand{\eqnref}[1]{(\ref{eq:#1})}
\newcommand{\figref}[1]{Fig.\ \ref{fig:#1}}
\newcommand{\Figref}[1]{Fig.\ \ref{fig:#1}}
\newcommand{\secref}[1]{Sec.\ \ref{sec:#1}}
\newcommand{\Tabref}[1]{Table \ref{tab:#1}}

\newcommand{\units}[1]{\ensuremath{~\mathrm{#1}}}

\newcommand{\sub}[1]{\ensuremath{_\text{#1}}}
\newcommand{\super}[1]{\ensuremath{^\text{#1}}}
\newcommand{\dd}{\ensuremath{\text{d}}}
\newcommand{\diff}[2]{\ensuremath{\frac{\dd {#1}}{\dd {#2}}}}
\newcommand{\difftwo}[2]{\ensuremath{\frac{\dd^2 {#1}}{\dd {#2}^2}}}
\newcommand{\partialdiff}[2]{\ensuremath{\frac{\partial {#1}}{\partial {#2}}}}
\newcommand{\intd}[4]{\ensuremath{\int_{#1}^{#2}{#3}\,\dd{#4}}}
\newcommand{\recip}[1]{\ensuremath{\frac{1}{#1}}}
\newcommand{\grad}{\ensuremath{\boldsymbol{\nabla}}}
\newcommand{\order}[1]{\ensuremath{\mathcal{O}({#1})}}

\begin{document}

%\preprint{}

\title{Erratum: Linearized $f(R)$ gravity: Gravitational radiation and Solar System tests \\ {[Phys.\ Rev.\ D 83, 104022 (2011)]}}

\author{Christopher P.L. Berry}
\email[]{cplb2@ast.cam.ac.uk}
\author{Jonathan R. Gair}
\email[]{jgair@ast.cam.ac.uk}
\affiliation{Institute of Astronomy, Madingley Road, Cambridge, CB3 0HA, United Kingdom}

\date{\today}

% 04.50.Kd 	Modified theories of gravity
% 04.25.Nx 	Post-Newtonian approximation; perturbation theory; related approximations
% 04.30.-w 	Gravitational waves
% 04.70.-s 	Physics of black holes
\pacs{04.50.Kd, 04.25.Nx, 04.30.--w, 04.70.--s}

\maketitle

In Sec. VIII A, (130) should read
\begin{equation}
g_{00} = 1 - 2U; \qquad g_{ij} = -(1 + 2\gamma U)\delta_{ij},
\end{equation}
with a minus sign in the expression for $g_{00}$. The expression for the post-Newtonian parameter $\gamma$ (147) should read
\begin{equation}
\gamma = -\frac{g_{00} + g_{ii}}{2U} - 1 \qquad \text{(no summation)},
\end{equation}
which includes an additional minus sign. Neither of these sign errors are propagated, and all results were based upon the correct forms for the equations. None of our conclusions, including that $\gamma = 1$, is modified.

\end{document}
